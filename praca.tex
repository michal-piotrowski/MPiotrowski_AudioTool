\documentclass[11pt]{aghdpl}
\usepackage[english,polish]{babel}
\usepackage{polski}
\usepackage[utf8]{inputenc}
\usepackage{mathtools}
\usepackage{amsfonts}
\usepackage{amsmath}
\usepackage{amsthm}

% --- < bibliografia > ---

\usepackage[
style=numeric,
sorting=none,
%
% Zastosuj styl wpisu bibliograficznego właściwy językowi publikacji.
language=autobib,
autolang=other,
% Zapisuj datę dostępu do strony WWW w formacie RRRR-MM-DD.
urldate=iso8601,
% Nie dodawaj numerów stron, na których występuje cytowanie.
backref=false,
% Podawaj ISBN.
isbn=true,
% Nie podawaj URL-i, o ile nie jest to konieczne.
url=false,
%
% Ustawienia związane z polskimi normami dla bibliografii.
maxbibnames=3,
% Jeżeli używamy BibTeXa:
backend=bibtex
]{biblatex}
\usepackage{csquotes}
% Ponieważ `csquotes` nie posiada polskiego stylu, można skorzystać z mocno zbliżonego stylu chorwackiego.
\DeclareQuoteAlias{croatian}{polish}
\addbibresource{bibliografia.bib}
% ------------------------
% --- < listingi > ---
% Użyj czcionki kroju Courier.
\usepackage{courier}

\usepackage{listings}
\lstloadlanguages{TeX}

\lstset{
	literate={ą}{{\k{a}}}1
           {ć}{{\'c}}1
           {ę}{{\k{e}}}1
           {ó}{{\'o}}1
           {ń}{{\'n}}1
           {ł}{{\l{}}}1
           {ś}{{\'s}}1
           {ź}{{\'z}}1
           {ż}{{\.z}}1
           {Ą}{{\k{A}}}1
           {Ć}{{\'C}}1
           {Ę}{{\k{E}}}1
           {Ó}{{\'O}}1
           {Ń}{{\'N}}1
           {Ł}{{\L{}}}1
           {Ś}{{\'S}}1
           {Ź}{{\'Z}}1
           {Ż}{{\.Z}}1,
	basicstyle=\footnotesize\ttfamily,
}
% ------------------------
\AtBeginDocument{
	\renewcommand{\tablename}{Tabela}
	\renewcommand{\figurename}{Rys.}
}
% ------------------------
% --- < tabele > ---

\usepackage{array}
\usepackage{tabularx}
\usepackage{multirow}
\usepackage{booktabs}
\usepackage{makecell}
\usepackage[flushleft]{threeparttable}

% defines the X column to use m (\parbox[c]) instead of p (`parbox[t]`)
\newcolumntype{C}[1]{>{\hsize=#1\hsize\centering\arraybackslash}X}
%---------------------------------------------------------------------------

\date{2015}

\setlength{\cftsecnumwidth}{10mm}

%---------------------------------------------------------------------------
\setcounter{secnumdepth}{4}
\brokenpenalty=10000\relax

\begin{document}

% Ponowne zdefiniowanie stylu `plain`, aby usunąć numer strony z pierwszej strony spisu treści i poszczególnych rozdziałów.
\fancypagestyle{plain}
{
	% Usuń nagłówek i stopkę
	\fancyhf{}
	% Usuń linie.
	\renewcommand{\headrulewidth}{0pt}
	\renewcommand{\footrulewidth}{0pt}
}

\setcounter{tocdepth}{2}
\graphicspath{ {images/} }
\clearpage
\title{
{\includegraphics[scale=0.8]{LogoAGH.jpg}  }\\
{\fontfamily{phv}\selectfont \large \textbf{WYDZIAŁ ELEKTRONIKI, AUTOMATYKI, INFORMATYKI I INŻYNIERII BIOMEDYCZNEJ \\
KATEDRA INFORMATYKI STOSOWANEJ}}\\
{Praca dyplomowa inżynierska} \\
{Aplikacja Webowa do przetwarzania ścieżek dźwiękowych}
}
\date{\today}
\author{Michał Piotrowski}

\maketitle
\thispagestyle{empty} 
{\large\itshape Uprzedzony o~odpowiedzialno\'sci karnej na~podstawie art.~115 ust.~1~i~2 ustawy z~dnia 4~lutego 1994~r. o~prawie autorskim i~prawach pokrewnych (t.j. Dz.U. z~2006~r. Nr~90, poz.~631 z~p{\'o}{\'z}n. zm.): ,,Kto przyw\l{}aszcza sobie autorstwo albo~wprowadza w~b\l{}\k{a}d co~do~autorstwa ca\l{}o\'sci lub cz\k{e}\'sci cudzego utworu albo~artystycznego wykonania, podlega grzywnie, karze ograniczenia wolno\'sci albo~pozbawienia wolno\'sci do~lat~3. Tej~samej karze podlega, kto~rozpowszechnia bez~podania nazwiska lub~pseudonimu tw\'orcy cudzy utw\'or w~wersji oryginalnej albo~w~postaci opracowania, artystycznego wykonania albo~publicznie zniekszta\l{}ca taki utw\'or, artystyczne wykonanie, fonogram, wideogram lub nadanie.'', a~tak\.ze uprzedzony o~odpowiedzialno\'sci dyscyplinarnej na~podstawie art.~211 ust.~1 ustawy z~dnia 27~lipca 2005~r. Prawo o~szkolnictwie wy\.zszym (t.j. Dz.~U. z~2012~r. poz.~572, z~p\'o\'zn. zm.): ,,Za~naruszenie przepis\'ow obowi\k{a}zuj\k{a}cych w~uczelni oraz~za~czyny uchybiaj\k{a}ce godno\'sci studenta student ponosi odpowiedzialno\'s\'c dyscyplinarn\k{a} przed komisj\k{a} dyscyplinarn\k{a} albo~przed s\k{a}dem kole\.ze\'nskim samorz\k{a}du studenckiego, zwanym dalej <<s\k{a}dem kole\.ze\'nskim>>.'', o\'swiadczam, \.ze~niniejsz\k{a} prac\k{e} dyplomow\k{a} wykona\l{}em(-am) osobi\'scie i~samodzielnie i~\.ze~nie~korzysta\l{}em(-am) ze~\'zr\'ode\l{} innych ni\.z~wymienione w pracy. }
\clearpage
\tableofcontents
\chapter{Wprowadzenie}
\section{Cel projektu} 
\setlength{\parindent}{10ex} \par Celem projektu jest odpowiedź na potrzebę stworzenia łatwo dostępnego i intuicyjnego w użytkowaniu systemu do modyfikowania ścieżek dźwiękowych. Wraz z postępem technologicznym i dostępnością narzędzi do przechwytywania dźwięku instrumentów, ludzkiego głosu i z innych źródeł pożądanym staje się posiadanie w zasięgu ręki narzędzia które nie ogranicza kreatywności muzycznej przez swoje skomplikowanie czy potrzebę instalowania środowiska o dużych wymaganiach co do zasobów obliczeniowych i pamięciowych, a co za tym idzie bycia przywiązanym do swojej stacji roboczej do wykonania nawet najprostrzych operacji na dźwięku


\chapter{Efekty dzwiekowe}

\par
Istotną częścią procesu tworzenia jest uwydatnienie najcenniejszych elementów nagranego dźwięku co jest umożliwione przez wypracowane przez lata algorytmy replikowanie analogowych efektów dźwiękowych w postaci cyfrowej 
\section{Echo}
\section{Pogłos}
\section{Kompresja}
\section{Korekcja Pasma (graficzna)}

%\include{rozdzial3}

\printbibliography

\end{document}
